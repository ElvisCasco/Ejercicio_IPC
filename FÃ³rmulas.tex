% Options for packages loaded elsewhere
\PassOptionsToPackage{unicode}{hyperref}
\PassOptionsToPackage{hyphens}{url}
\PassOptionsToPackage{dvipsnames,svgnames,x11names}{xcolor}
%
\documentclass[
  landscape]{report}

\usepackage{amsmath,amssymb}
\usepackage{iftex}
\ifPDFTeX
  \usepackage[T1]{fontenc}
  \usepackage[utf8]{inputenc}
  \usepackage{textcomp} % provide euro and other symbols
\else % if luatex or xetex
  \usepackage{unicode-math}
  \defaultfontfeatures{Scale=MatchLowercase}
  \defaultfontfeatures[\rmfamily]{Ligatures=TeX,Scale=1}
\fi
\usepackage{lmodern}
\ifPDFTeX\else  
    % xetex/luatex font selection
\fi
% Use upquote if available, for straight quotes in verbatim environments
\IfFileExists{upquote.sty}{\usepackage{upquote}}{}
\IfFileExists{microtype.sty}{% use microtype if available
  \usepackage[]{microtype}
  \UseMicrotypeSet[protrusion]{basicmath} % disable protrusion for tt fonts
}{}
\makeatletter
\@ifundefined{KOMAClassName}{% if non-KOMA class
  \IfFileExists{parskip.sty}{%
    \usepackage{parskip}
  }{% else
    \setlength{\parindent}{0pt}
    \setlength{\parskip}{6pt plus 2pt minus 1pt}}
}{% if KOMA class
  \KOMAoptions{parskip=half}}
\makeatother
\usepackage{xcolor}
\usepackage[top=5mm,left=10mm,right=10mm]{geometry}
\setlength{\emergencystretch}{3em} % prevent overfull lines
\setcounter{secnumdepth}{-\maxdimen} % remove section numbering
% Make \paragraph and \subparagraph free-standing
\makeatletter
\ifx\paragraph\undefined\else
  \let\oldparagraph\paragraph
  \renewcommand{\paragraph}{
    \@ifstar
      \xxxParagraphStar
      \xxxParagraphNoStar
  }
  \newcommand{\xxxParagraphStar}[1]{\oldparagraph*{#1}\mbox{}}
  \newcommand{\xxxParagraphNoStar}[1]{\oldparagraph{#1}\mbox{}}
\fi
\ifx\subparagraph\undefined\else
  \let\oldsubparagraph\subparagraph
  \renewcommand{\subparagraph}{
    \@ifstar
      \xxxSubParagraphStar
      \xxxSubParagraphNoStar
  }
  \newcommand{\xxxSubParagraphStar}[1]{\oldsubparagraph*{#1}\mbox{}}
  \newcommand{\xxxSubParagraphNoStar}[1]{\oldsubparagraph{#1}\mbox{}}
\fi
\makeatother


\providecommand{\tightlist}{%
  \setlength{\itemsep}{0pt}\setlength{\parskip}{0pt}}\usepackage{longtable,booktabs,array}
\usepackage{calc} % for calculating minipage widths
% Correct order of tables after \paragraph or \subparagraph
\usepackage{etoolbox}
\makeatletter
\patchcmd\longtable{\par}{\if@noskipsec\mbox{}\fi\par}{}{}
\makeatother
% Allow footnotes in longtable head/foot
\IfFileExists{footnotehyper.sty}{\usepackage{footnotehyper}}{\usepackage{footnote}}
\makesavenoteenv{longtable}
\usepackage{graphicx}
\makeatletter
\newsavebox\pandoc@box
\newcommand*\pandocbounded[1]{% scales image to fit in text height/width
  \sbox\pandoc@box{#1}%
  \Gscale@div\@tempa{\textheight}{\dimexpr\ht\pandoc@box+\dp\pandoc@box\relax}%
  \Gscale@div\@tempb{\linewidth}{\wd\pandoc@box}%
  \ifdim\@tempb\p@<\@tempa\p@\let\@tempa\@tempb\fi% select the smaller of both
  \ifdim\@tempa\p@<\p@\scalebox{\@tempa}{\usebox\pandoc@box}%
  \else\usebox{\pandoc@box}%
  \fi%
}
% Set default figure placement to htbp
\def\fps@figure{htbp}
\makeatother

\makeatletter
\@ifpackageloaded{caption}{}{\usepackage{caption}}
\AtBeginDocument{%
\ifdefined\contentsname
  \renewcommand*\contentsname{Table of contents}
\else
  \newcommand\contentsname{Table of contents}
\fi
\ifdefined\listfigurename
  \renewcommand*\listfigurename{List of Figures}
\else
  \newcommand\listfigurename{List of Figures}
\fi
\ifdefined\listtablename
  \renewcommand*\listtablename{List of Tables}
\else
  \newcommand\listtablename{List of Tables}
\fi
\ifdefined\figurename
  \renewcommand*\figurename{Gráfico}
\else
  \newcommand\figurename{Gráfico}
\fi
\ifdefined\tablename
  \renewcommand*\tablename{Tabla}
\else
  \newcommand\tablename{Tabla}
\fi
}
\@ifpackageloaded{float}{}{\usepackage{float}}
\floatstyle{ruled}
\@ifundefined{c@chapter}{\newfloat{codelisting}{h}{lop}}{\newfloat{codelisting}{h}{lop}[chapter]}
\floatname{codelisting}{Listing}
\newcommand*\listoflistings{\listof{codelisting}{List of Listings}}
\makeatother
\makeatletter
\makeatother
\makeatletter
\@ifpackageloaded{caption}{}{\usepackage{caption}}
\@ifpackageloaded{subcaption}{}{\usepackage{subcaption}}
\makeatother

\usepackage{bookmark}

\IfFileExists{xurl.sty}{\usepackage{xurl}}{} % add URL line breaks if available
\urlstyle{same} % disable monospaced font for URLs
\hypersetup{
  pdftitle={Fórmulas},
  pdfauthor={Elvis Casco},
  colorlinks=true,
  linkcolor={blue},
  filecolor={Maroon},
  citecolor={Blue},
  urlcolor={Blue},
  pdfcreator={LaTeX via pandoc}}


\title{Fórmulas}
\author{Elvis Casco}
\date{}

\begin{document}
\maketitle


\chapter{IPC por Elementos}\label{ipc-por-elementos}

Esta explicación se toma del Manual, para explicar a partir de los
elementos mas pequeños: variedades. En dicho manual, a estos se les
nombra como \textbf{agregados elementales}.

Los agregados elementales son grupos de bienes y servicios relativamente
homogéneos, que pueden abarcar todo el país o solo regiones
individuales. Asimismo, pueden establecerse distintos agregados
elementales para distintos tipos de puntos de venta. L

Para su escogencia se tienen en cuenta los siguientes elementos:

\begin{itemize}
\tightlist
\item
  Los agregados elementales deberían componerse de grupos de bienes o
  servicios tan parecidos entre sí como sea posible y, preferentemente,
  homogéneos.
\item
  Deberían estar compuestos de artículos de los cuales se esperan
  variaciones de precios parecidas, a efectos de minimizar la dispersión
  de las variaciones de precios dentro del agregado.
\item
  Los agregados elementales deberían ser apropiados para servir como
  estratos para propósito de muestreo en función del régimen de muestreo
  que se establezca para la recopilación de datos.
\end{itemize}

Utilizando una clasificación de los gastos del consumidor como la
Clasificación del Consumo Individual por Finalidades (CCIF), todo el
conjunto de bienes y servicios de consumo que abarca el IPC nivel
general puede dividirse en grupos, por ejemplo ``comestibles y bebidas
no alcohólicas''. Cada grupo se divide a su vez en clases, por ejemplo,
``comestibles''. A los fines del IPC, cada clase puede dividirse a su
vez en subclases más homogéneas, como ``arroz''. Las subclases equivalen
a los capítulos del Programa de Comparación Internacional, que calcula
las paridades de poder adquisitivo (PPA) entre países.

\chapter{Índices}\label{uxedndices}

\chapter{Lowe:}\label{lowe}

El período cuyas cantidades efectivamente se utilizan en el IPC se
conoce como \emph{período de referencia de las ponderaciones}, y se
denotará como período \(b\). El período 0 es el período de referencia de
los precios.

Sea \(n\) la cantidad de productos en una canasta con precios \(p_i\) y
cantidades \(q_i\), y sean 0 y \(t\) los dos períodos que se comparan.

El índice de Lowe \(P_{Lo}\) para el producto \(i\) de la región \(r\)
se define de la siguiente manera:

\(P_{Lo}=\frac{\sum_{i=1}^n p_i^t q_i}{\sum_{i=1}^n p_0^t q_i}\)

El índice de Lowe que utiliza las cantidades del período \(b\) puede
expresarse de la siguiente forma:

\(P_{Lo}=\frac{\sum_{i=1}^n p_i^t q_i^b}{\sum_{i=1}^n p_i^0 q_i^b}\)

\(P_{Lo}=\sum_{i=1}^n \frac{p_i^t}{p_i^0} s_i^{0b}\)

donde

\(s_i^{0b}=\frac{p_i^0q_i^b}{\sum_{i=1}^n p_i^0q_i^b}\)

\begin{enumerate}
\def\labelenumi{\arabic{enumi}.}
\setcounter{enumi}{1}
\tightlist
\item
  Índice de Precios por Establecimiento (e) y Variedad (V)
\end{enumerate}

\(i_{ev}^r = \frac{\frac{p_{ev,t}^r}{c_{ev,t}^r}}{\frac{p_{ev,t-1}^r}{c_{ev,t-1}^r}}\)

\begin{enumerate}
\def\labelenumi{\arabic{enumi}.}
\setcounter{enumi}{2}
\tightlist
\item
  Índice de Precios por Variedad: Media Geométrica
\end{enumerate}

\(i_V^r = \sqrt[n]{{i_{ev_1}^r,i_{ev_2}^r,...,i_{ev_n}^r}} \text{ for v in } V^r\)

\(i_V^r = \exp (\frac{{\ln i_{ev_1}^r + \ln i_{ev_2}^r +...+\ln i_{ev_n}^r}}{n}) \text{ for v in } V^r\)

\begin{enumerate}
\def\labelenumi{\arabic{enumi}.}
\setcounter{enumi}{3}
\tightlist
\item
  Índice de Precios por Producto (X): Media Geométrica
\end{enumerate}

\(i_X^r = \sqrt[n]{{i_{V_1}^r,i_{V_2}^r,...,i_{V_n}^r}} \text{ for V in } X^r\)

\(i_X^r = \exp (\frac{{\ln i_{V_1}^r + \ln i_{V_2}^r +...+\ln i_{V_n}^r}}{n}) \text{ for V in } X^r\)

\(Indice\_Pond^r\) = \(i_X^r * w_X^r\)

\(IPC^r\) = \(\sum{Indice\_Pond^r}\)

\(w_A^r = \frac{w_X^r}{\sum w_X^r} \text{ for X in } A^r\)

\(i_A^r = \sum i_X^r * \frac{w_X^r}{w_A^r} \text{ for X in } A^r\)




\end{document}
